\documentclass[a4paper, 11pt]{article}
\usepackage[utf8]{inputenc} 
\usepackage[T1]{fontenc}
\usepackage{lmodern}
\usepackage[french]{babel}
\usepackage{fullpage}
 
\begin{document}
 
\title{Projet de compilation}
\author{Nguy\~{\^e}n Lê Thành D\~ung \and Nicolas ``Koliaza'' Blanchard}
\date{Novembre 2012}
\maketitle

\section{Présentation générale}

\subsection{Avancement du projet}

Pour l'instant, le lexer, le parser et le typeur ont été écrits, ce dernier
ne faisant que vérifier que le programme est bien typé, sans produire l'arbre
de syntaxe désucré et annoté qui servira de support à la production de code.

\subsection{Compiler et exécuter}

Le projet est muni d'un Makefile qui propose les cibles suivantes :
\begin{description}
\item[minic] la cible par défaut ; taper simplement \texttt{make} en ligne
  de commande permet de produire l'exécutable \texttt{minic} du compilateur
\item[tests] compile le compilateur si besoin est, puis lance une suite de tests
\item[clean] nettoie le dossier du projet pour ne laisser que les sources
\end{description}




\section{Choix techniques}

\subsection{Analyse lexicale}

Rather straightforward section.
Parler de l'approche utilisée pour gérer les commentaires/whitespace avec
des règles plutôt que des regex.
% + Lexing.new_line
+ gestion des littéraux

\subsection{Analyse syntaxique}

Grammaire à peu près copiée sur la spécification du projet.
Utilisation des fonctionnalités de Menhir :
% règle paramétrée with_location, usage de inline
message d'erreur pas très explicite (``Syntax error'')
token error --> compliqué
Dépendance sur le bon comportement de Menhir pour la résolution de conflits...
côté ``black box'' insatisfaisant.
Complications liées à la gestion des déclarations de variables
int *a, b, **c;
Convention de nommage : token Plus (nom symbole) -> binop Add (interprétation)
Proximité avec la syntaxe concrète pour aider la signalisation d'erreur.
Désucrage réduit ; choix fait lors de l'annotation de la localisation
d'un noeud provenant d'un désucrage.

\subsection{Vérification du typage}

Code lourd, moche et bureaucratique !!!
3 grosses fonctions, et une grosse déclaration de type.
Usage important du pattern matching, et de petites fonctions pour éviter les redondances.
%Gestion des erreurs, maybe_find, et passage des localisations partout.
--> pourquoi pas de Map.find avec type option ?
--> définition de fonctions locales currifiées dans chaque fonction pour
  économiser de l'écriture
Bénéfices potentiels des variables dynamically scoped.

\subsection{Interaction utilisateur}

\subsubsection{Point d'entrée du programme}

main.ml
utilise le module Arg --> option --help gratuite
point central de gestion des exceptions
affichage des messages d'erreur
--> format de signalisation restrictif (pour le typage)

\subsubsection{Suite de tests}

Programme OCaml utilisant le module Unix pour exécuter minic sur tous les fichiers .c
dans le dossier tests ; affiche les résultats sur stdout.
Tous les tests ont l'air de fonctionner.
Prévu pour Linux.

\subsubsection{Chaîne de compilation}

Ocamlbuild caché derrière make : marche sous Linux.
Cibles minic, tests et clean. (mrproper --> useless)
Sous Windows, difficultés rencontrées pour faire fonctionner Ocaml.


\end{document}