\documentclass[a4paper, 11pt]{article}
\usepackage[utf8]{inputenc} 
\usepackage[T1]{fontenc}
\usepackage{lmodern}
\usepackage[french]{babel}
\usepackage{fullpage}
\usepackage{listings}
 
\begin{document}
 
\title{Projet de compilation}
\author{Nguy\~{\^e}n Lê Thành D\~ung \and Nicolas Koliaza Blanchard}
\date{Novembre 2012}
\maketitle

\section{Présentation générale}

\subsection{Avancement du projet}

Pour l'instant, le lexer, le parser et le typeur ont été écrits. La version
actuelle du typeur ne génère pas d'AST décoré, elle se contente de vérifier
la correction du typage, l'unicité des définitions globales, etc. et renvoie
\texttt{unit}.

Le programme a été testé sur l'ensemble des fichiers de tests fournis, et
se comporte dessus comme attendu.

\subsection{Compiler et exécuter}

Le projet est muni d'un Makefile qui propose les cibles suivantes :
\begin{description}
\item[minic] la cible par défaut ; taper simplement \texttt{make} en ligne
  de commande permet de produire l'exécutable \texttt{minic} du compilateur
\item[tests] compile le compilateur si besoin est, puis lance une suite de tests
\item[clean] nettoie le dossier du projet pour ne laisser que les sources
\end{description}

Une fois le programme compilé, la commande \texttt{./minic [options] fichier.c}
exécute les phases de la compilation sur le contenu du fichier \texttt{fichier.c}.
Pour l'instant, les options suivantes sont disponibles :
\begin{description}
\item[--help] affiche la liste des options
\item[-parse-only] s'arrête à l'analyse syntaxique
\item[-type-only] s'arrête à l'analyse sémantique : c'est en fait le comportement
  par défaut, car la production de code n'est pas encore réalisée
\item[--version] sert à faire joli :-)
\end{description}


\section{Choix techniques}

\subsection{Considérations globales}

Le projet a été réalisé avec un choix d'outils de programmation plutôt conventionnel :
OCaml/OCamllex/Menhir. Il s'est inspiré d'exemples donnés en cours et en TD, notamment
le parser mini-Pascal.

\subsubsection{Structure du programme}

Le compilateur a été découpé en les grandes phases suivantes :
\begin{itemize}
\item Analyse lexicale
\item Analyse syntaxique
\item Typage / analyse sémantique
\item Production de code
\end{itemize}

Chacun de ces modules est contenu dans un fichier \texttt{.ml}, avec de plus
un fichier \texttt{.mli} permettant de restreindre les fonctions et types
visibles depuis l'extérieur (cas particulier : le fichier d'interface du parser
est généré automatiquement par Menhir).

Il y a de plus un fichier \texttt{main.ml} contenant le point d'entrée du programme :
il s'occupe d'effectuer le lien entre les fonctions contenues dans ces modules,
afin de pouvoir véritablement compiler un fichier source fourni par l'utilisateur.

\subsubsection{Chaîne de compilation}

Si l'usage du Makefile permet de présenter une interface familière aux
utilisateurs de systèmes Unix voulant le compiler, il ne fait en fait que cacher
des appels à Ocamlbuild, un système de build qui calcule les dépendances
entre fichiers sources OCaml et les compile dans l'ordre, et peut appeler automatiquement
Ocamllex et Menhir. Ainsi le Makefile est très simple, il ne fait que 11 lignes !

\subsection{Analyse lexicale}

Le lexer est généré par Ocamllex à partir du fichier \texttt{lexer.mll}.

L'écriture du lexer n'a pas en général posé de problème, c'était la section la plus
simple du projet, comportant une certaine dose de répétitivité (il y a 11 mots-clé à
gérer dans le Mini-C\ldots).

Une subtilité cependant dans le fonctionnement de Ocamllex : il faut explicitement lui
indiquer les changements de ligne en appelant la fonction \texttt{Lexing.new\_line} sur
le buffer. Sinon, les informations de position données par \texttt{Lexing.lexeme\_start\_p},
ou encore par \texttt{\$startpos} et \texttt{\$endpos} dans Menhir, n'indiqueront pas
le bon numéro de ligne. Ce problème nous a forcé à abandonner une première approche pour
la gestion des blancs et des commentaires basée sur une reconnaissance par expression
rationnelle et élimination directe ; nous avons dû les traiter en rajoutant des règles
dans le lexer, pour obtenir un comportement correct à la lecture de \texttt{'\textbackslash n'}.
Notons le gain de lisibilité obtenu par rapport à la regex des commentaires que nous avions :
\begin{lstlisting}[language=Caml]
let comment = "//" [^ '\n']*
            | "/*" ([^ '*'] | ('*'* [^ '/' '*']))* '*'* "*/"
\end{lstlisting}

Pour gérer les constantes littérales, nous avons choisi de récupérer d'abord la chaîne
de caractères les représentant grâce à une regex, puis de déchiffrer cette chaîne grâce
à des fonctions auxiliaires. Dans le cas des entiers, la fonction de lecture est en fait
une règle du lexer que l'on applique non pas à \texttt{lexbuf}, mais à un buffer créé
à partir de la chaîne.

Dernière note : les littéraux entiers sont stockés comme des Int32, afin de pouvoir
représenter toute la plage de valeurs disponibles pour un mot machine sur un MIPS 32 bits,
qui sera l'architecture cible du compilateur.

\subsection{Analyse syntaxique et arbre de syntaxe}

Le parser est généré par Menhir à partir du fichier \texttt{parser.mly}.

Les règles de grammaire ont été dans un premier temps copiées à partir de la
spécification du projet, ce qui a permis d'obtenir rapidement (après quelques modifications)
un squelette à compléter avec les actions produisant l'AST.

Nous avons fait un usage important des fonctionnalités de Menhir afin d'améliorer la
concision du code. Citons notamment :
\begin{itemize}
\item les règles \texttt{\%inline}, permettant notamment de gérer les opérateurs binaires
  avec les bonnes priorités
\item l'usage des règles paramétrées : par exemple, parser la liste des arguments
  d'une fonction est facilité par la règle \texttt{separated\_list}
\item lorsqu'il a fallu implémenter les informations de localisation dans l'arbre
  de syntaxe, les mots-clé \texttt{\$startpos} et \texttt{\$endpos} ainsi que la
  possibilité de définir notre propre règle paramétrée \texttt{with\_location}
  ont permis de faire fonctionner le parser avec un minimum de modifications
\end{itemize}

L'arbre de syntaxe a été spécifié de façon délibérément proche de la syntaxe concrète,
afin de faciliter la signalisation des erreurs lors de la phase suivante ; ainsi, les boucles
\texttt{for} apparaissent comme le noeud \texttt{Ast.For} plutôt que d'être immédiatement
désucrées en boucles \texttt{while}. Pour la même raison, l'AST est annoté de couples
de \texttt{Lexing.position} indiquant l'emplacement (début et fin) de chaque syntagme
dans le fichier.

Ont été désucrés : \texttt{a->b} en \texttt{(*a).b} et \texttt{a[b]} en \texttt{*(a+b)}.
Dans ces cas, nous avons décidé d'indiquer comme localisation des noeuds intermédiaires
introduits l'emplacement du syntagme original entier.

message d'erreur pas très explicite (``Syntax error'')
token error --> compliqué
Dépendance sur le bon comportement de Menhir pour la résolution de conflits...
côté ``black box'' insatisfaisant.
Complications liées à la gestion des déclarations de variables
int *a, b, **c;
Convention de nommage : token Plus (nom symbole) -> binop Add (interprétation)
Proximité avec la syntaxe concrète pour aider la signalisation d'erreur.
Désucrage réduit ; choix fait lors de l'annotation de la localisation
d'un noeud provenant d'un désucrage.

\subsection{Vérification du typage}

Code lourd, moche et bureaucratique !!!
3 grosses fonctions, et une grosse déclaration de type.
Usage important du pattern matching, et de petites fonctions pour éviter les redondances.
%Gestion des erreurs, maybe_find, et passage des localisations partout.
--> pourquoi pas de Map.find avec type option ?
--> définition de fonctions locales currifiées dans chaque fonction pour
  économiser de l'écriture
Bénéfices potentiels des variables dynamically scoped.
Poids en lignes de code des messages d'erreur...

\subsection{Interaction utilisateur}

\subsubsection{Point d'entrée du programme}

main.ml
utilise le module Arg --> option --help gratuite
point central de gestion des exceptions
affichage des messages d'erreur
--> format de signalisation restrictif (pour le typage)

\subsubsection{Suite de tests}

Programme OCaml utilisant le module Unix pour exécuter minic sur tous les fichiers .c
dans le dossier tests ; affiche les résultats sur stdout.
Tous les tests ont l'air de fonctionner.
Prévu pour Linux.



\end{document}