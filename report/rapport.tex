\documentclass[a4paper, 11pt]{article}
\usepackage[utf8]{inputenc} 
\usepackage[T1]{fontenc}
\usepackage{lmodern}
\usepackage[french]{babel}
\usepackage{fullpage}
\usepackage{listings}
\usepackage{hyperref}
 
\begin{document}
 
\title{Projet de compilation}
\author{Nguy\~{\^e}n Lê Thành D\~ung \and Nicolas Koliaza Blanchard}
\date{Novembre 2012}
\maketitle

\section{Présentation générale}

\subsection{Avancement du projet}

Pour l'instant, le lexer, le parser et le typeur ont été écrits. La version
actuelle du typeur ne génère pas d'AST décoré, elle se contente de vérifier
la correction du typage, l'unicité des définitions globales, etc. et renvoie
\texttt{unit}.

Le programme a été testé sur l'ensemble des fichiers de tests fournis, et
se comporte dessus comme attendu.

\subsection{Compiler et exécuter}

Le projet est muni d'un Makefile qui propose les cibles suivantes :
\begin{description}
\item[minic] la cible par défaut ; taper simplement \texttt{make} en ligne
  de commande permet de produire l'exécutable \texttt{minic} du compilateur
\item[tests] compile le compilateur si besoin est, puis lance une suite de tests
\item[clean] nettoie le dossier du projet pour ne laisser que les sources
\end{description}

Une fois le programme compilé, la commande \texttt{./minic [options] fichier.c}
exécute les phases de la compilation sur le contenu du fichier \texttt{fichier.c}.
Pour l'instant, les options suivantes sont disponibles :
\begin{description}
\item[--help] affiche la liste des options
\item[-parse-only] s'arrête à l'analyse syntaxique
\item[-type-only] s'arrête à l'analyse sémantique : c'est en fait le comportement
  par défaut, car la production de code n'est pas encore réalisée
\item[--version] sert à faire joli :-)
\end{description}

\subsection{Dépôt git}

Le code source est disponible sur Github à l'adresse \url{http://github.com/nguyentito/compilatateur}.
Tous les commits ont été faits depuis la même machine, ce qui ne reflète pas le partage réel du travail. On y trouvera le brouillon d'un rapport plus complet que nous avons dû abréger.

\section{Choix techniques}

\subsection{Considérations globales}

\subsubsection{Structure du programme}

Le compilateur a été découpé en les grandes phases suivantes :
\begin{itemize}
\item Analyse lexicale
\item Analyse syntaxique
\item Typage / analyse sémantique
\item Production de code
\end{itemize}

\subsubsection{Chaîne de compilation}

On a un Makefile qui fait appel à OCamlbuild.

\subsection{Analyse lexicale}

L'écriture du lexer était relativement simple bien qu'un peu répétitive.

Quelques remarques :
\begin{itemize}
\item il a fallu utiliser \texttt{Lexing.new\_line} pour avoir une gestion correcte
  des numéros de ligne dans les localisations
\item on avait d'abord utilisé des regex pour éliminer les blancs et les commentaires,
  mais il a fallu utiliser des règles du lexer à la place (et c'est plus lisible)
\item les littéraux int/char sont stockés comme des Int32
\end{itemize}
Nous ne résistons pas au plaisir de vous montrer la défunte regex :
\begin{lstlisting}[language=Caml]
let comment = "//" [^ '\n']*
            | "/*" ([^ '*'] | ('*'* [^ '/' '*']))* '*'* "*/"
\end{lstlisting}

\subsection{Analyse syntaxique et arbre de syntaxe}

Les règles de grammaire ont été dans un premier temps copiées à partir de la
spécification du projet, ce qui a permis d'obtenir rapidement (après quelques modifications)
un squelette à compléter avec les actions produisant l'AST.

L'arbre de syntaxe est proche de la syntaxe concrète, avec un minimum de désucrage.
Ca facilite la signalisation des erreurs pour la suite.

Cependant certaines choses ne sont toujours pas satisfaisantes : 
\begin{itemize}
\item Le fait que les messages d'erreur (``Syntax error'') ne soient pas explicites.
\item La dépendance sur Menhir et son côté boîte noire.
\item La gestion des déclarations avec multi-pointeurs qui n'est pas élégante.
\end{itemize}


\subsection{Vérification du typage}

C'est la partie lourde du code, avec près de 500 lignes, principalement pour les listes d'erreurs, même en essayant d'éviter les répétitions. 
Deux questions subsiste :
\begin{itemize}
\item dans les spécifications du projet la relation de compatibilité de type n'a pas
  à commuter avec les pointeurs ce qui semble pourtant logique. Ainsi \texttt{int*}
  $\not\equiv$ \texttt{char*}. L'implémenation actuelle suit ces spécifications.
\item Il y a des ambiguïtés liées au typage des expressions arithmétiques, ainsi
  0+0 pourrait aussi bien être de type \texttt{int} que de type \texttt{void*}.
\end{itemize}

\subsection{Point d'entrée du programme}

C'est dans le fichier \texttt{main.ml} que tout est centralisé, qu'on connecte
les différents modules, et qu'on gère l'affichage des erreurs qu'ils lèvent.


\subsection{Suite de tests}

Programme OCaml utilisant le module Unix pour exécuter minic sur tous les fichiers .c
dans le dossier tests ; affiche les résultats sur stdout.
Tous les tests ont l'air de fonctionner.
Prévu pour Linux.



\end{document}.